% !TEX root = numb3rs.tex
\newpage
\subsection{216: Protest}\label{216}

The only significant mathematical content in this episode was a brief discussion of social networking. We'll talk about an example of an experiment involving social networking and of a mathematical method to identifying important nodes in social networks. \\

\temph{Social Networks}

The basic idea of social networks is to represent the interactions between a group of people by a mathematical structure called a \emph{graph}. A graph is given by a set of \emph{vertices} (also called nodes), and a set of pairs of vertices, called \emph{edges}. To model friendships (or other types of interactions) among a group of people, we have one node for each person and an edge between each pair of people who are friends. Now we can talk about various quantities related to the graph or particular vertices. For example, we can count the number of edges at each vertex, the distance between two different vertices, etc. Here is an example graph:












































